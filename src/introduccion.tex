\chapter{INTRODUCCIÓN}

\clearpage

Internet ha tenido una evolución radical desde sus inicios. Hoy en dá, la web tiende a ser un espacio interactivo donde la comunicación cliente-servidor es bidireccional y la información se mueve en \textbf{tiempo real}.

Este cambio puede reflejarse en los teléfonos móviles, con la explosión de las apps de mensajería instantánea. Estas apps permiten el envío de mensajes de texto gratuitos a través de Internet. También ofrecen opciones de voz y video, y la posibilidad de compartir archivos.

Pero las apps que permiten comunicarnos con los servicios de emergencias no han evolucionado con la misma rapidez que la tecnología en tiempo real. Estas apps hasta ahora, solo permiten enviar de forma automática la ubicación actual a las centros de emergencias y, si es necesario, enviar fotografías para agilizar una primera intervención.

\section{Motivaciones del proyecto}

% \begin{figure}[htp!]
%   \centering
%   \includegraphics[width=0.75\textwidth,clip=true]{logo_politecnica}
%   \caption{Logo de la Universidad Politécnica de madrid.}
%   \label{fig:logo_uam}
% \end{figure}

El mundo del desarrollo web avanza rápido, cada día surgen nuevas herramientas y nuevas tendencias tecnológicas que debemos implementar si queremos seguir siendo competitivos dentro del mercado. El presente de las aplicaciones web, se podría decir que es la respuesta inmediata, lo que en tecnología se conoce como \textbf{tiempo real} \cite{nodejs10}.

Estos últimos años, las empresas han apostado por remodelar sus servicios para aprovecharse de los beneficios de integrar las tecnologías en tiempo real. Esto se debe a que ahora existen herramientas que nos facilitan enormemente este proceso que antes suponía un gran esfuerzo \cite{app7}.

Node.js es una de estas herramientas que en los últimos tiempos ha alcanzado una popularidad innegable, hasta llegar a ser un componente indispensable en el desarrollo de aplicaciones web en tiempo real.

Por otra parte, la mensajería instantánea, que ha evolucionado desde los años 90, hoy en día se ha sofisticado y adoptado como parte del uso cotidiano. Las compañías, las organizaciones políticas y otras entidades están utilizando cada vez más la mensajería instantánea como medio para comunicarse con los clientes e incluso, entre compañeros de trabajo.

Pero los centros 112 de Europa todavía no se han aprovechado de estas nuevas tendencias tecnológicas. Las apps de emergencias, de momento, solo ofrecen la posibilidad de comunicarte con los servicios de emergencias a través de llamadas de voz.

No obstante, existen proyectos que, a día de hoy, tratan de mejorar la respuesta de los servicios de emergencias, buscando tecnologías novedosas que sean fáciles de integrar en una primera fase. Teniendo en cuenta esto, es la oportunidad perfecta para mejorar la conectividad con los servicios de emergencias con una nueva alternativa de comunicación en tiempo real.

\section{Objetivos del proyecto}

Los ciudadanos de ahora usan todos los días comunicaciones basadas en IP, como la mensajería instantánea, y sería interesante la posibilidad de comunicarse con los servicios de emergencias utilizando estos medios. Pero la organización del 112 y los servicios de emergencias actuales solo son accesibles mediante llamadas telefónicas de voz.

Esta tendencia es motivo más que suficiente para que los PSAPs cambien sus plataformas tecnológicas y de este modo formar parte del futuro; es decir, mejorar sus sistemas de emergencias con la integración de las tecnologías en tiempo real.

De este nodo, los PSAPs puedan recibir no solo voz, sino también información de ubicación, texto en tiempo real, ficheros multimedia, videollamadas u otra información del llamante.

Por esta razón, el trabajo plantea el estudio de las nuevas soluciones y tecnologías disponibles que puedan mejorar los sistemas de emergencias. Y posteriormente, llevar a cabo el desarrollo de una solución software en tiempo real fácil de integrar.

No obstante, al tratarse de un nuevo servicio en los sistemas de emergencias es de vital importancia realizar un análisis del rendimiento y el coste del mismo teniendo en cuenta los requisitos específicos del 112.
