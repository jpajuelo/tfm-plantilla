\chapter{INTRODUCCIÓN}

Internet ha sufrido una evolución radical desde sus inicios. A día de hoy, la Web tiende a ser un espacio interactivo en el cual la comunicación entre un cliente y un servidor web es bidireccional y la información se mueve en tiempo real.

Este gran cambio puede reflejarse en los teléfonos móviles, con la explosión de las apps de mensajería instantánea. Estas apps permiten el envío de mensajes de texto gratuitos a través de Internet. También ofrecen opciones de voz y video, y la posibilidad de compartir archivos.

Pero las apps que permiten comunicarnos con los servicios de emergencias no han evolucionado con la misma rapidez que la tecnología en tiempo real. Estas apps hasta ahora, solo permiten enviar de forma automática la ubicación actual a las centros de emergencias y, si es necesario, enviar fotografías para agilizar una primera intervención.

\clearpage

\section{Motivaciones del proyecto}

El mundo del desarrollo web avanza rápido, cada día surgen nuevas herramientas y nuevas tendencias que debemos implementar si queremos seguir siendo competitivos dentro del mercado. El presente de las aplicaciones web, se podría afirmar que es la respuesta inmediata, lo que en tecnología se conoce como tiempo real.

Estos últimos años, las empresas han apostado por remodelar sus productos para aprovecharse de los beneficios de las aplicaciones web en tiempo real. Esto se debe a que integrar la funcionalidad de tiempo real supone un gran esfuerzo, pero ahora existen herramientas que nos facilitan enormemente este proceso.

Node.js es una de estas herramientas que en los últimos tiempos ha alcanzado una popularidad innegable, hasta llegar a ser un componente indispensable en el desarrollo de aplicaciones web.

Por otra parte, la mensajería instantánea, que ha evolucionado desde los años 90, hoy en día se ha sofisticado y adoptado como parte del uso cotidiano. Las compañías, las organizaciones políticas y otras entidades están utilizando cada vez más la mensajería instantánea como medio para comunicarse con los clientes e incluso, entre compañeros de trabajo.

Pero los centros 112 de Europa todavía no se han aprovechado de estas nuevas tendencias que han revolucionado el Internet. Las apps de emergencias, solo ofrecen la posibilidad de comunicarte con los servicios de emergencias a través de llamadas de voz.

Siendo conscientes de esto, algunos centros de emergencias, buscan incluir aquellas novedades tecnológicas que sean fáciles de incorporar a corto plazo. Con el objetivo de ofrecer una mejor respuesta de los servicios de emergencias y una nueva alternativa de comunicación en tiempo real.

\section{Objetivos del proyecto}

A día de hoy, los ciudadanos usan todos los días comunicaciones basadas en IP, como la mensajería instantánea, y sería interesante la posibilidad de comunicarse con los servicios de emergencias utilizando estos medios. Pero la organización del 112 y los servicios de emergencias solo son accesibles mediante llamadas telefónicas de voz.

Motivo más que suficiente para que los PSAPs cambien su tecnología para ser parte del futuro; es decir, mejorar sus sistemas de emergencias con la integración de las tecnologías en tiempo real, para recibir no solo voz, sino también información de ubicación, texto en tiempo real, ficheros multimedia o videollamadas.

Por esta razón, el trabajo plantea el estudio de las nuevas soluciones y tecnologías disponibles que puedan mejorar los sistemas de emergencias. Y posteriormente, llevar a cabo el desarrollo de un solución software en tiempo real con un baja complejidad de integración. No obstante, al tratarse de un nuevo servicio en los sistemas de emergencias es de vital importancia realizar un análisis del rendimiento y el coste del mismo.
