% Resumen en español
\chapter*{Resumen}

\begin{abstractEs}
	Hoy en día, la web tiende a ser un espacio interactivo donde la comunicación entre clientes y un servidor web es bidireccional y la información se mueve en “tiempo real”. Este cambio puede reflejarse en los teléfonos móviles, con el auge de las apps de mensajería instantánea que ofrecen una nueva forma de comunicación a través del envío de mensajes de texto gratuitos.

	Muchas empresas han apostado por remodelar sus servicios para aprovecharse de los beneficios de las tecnologías en tiempo real, debido a que ahora existen herramientas que nos facilitan enormemente este proceso que antes suponía un gran esfuerzo.

	Sin embargo, los centros 112 (PSAPs) de toda Europa todavía no se han aprovechado de estas nuevas tendencias tecnológicas. Hasta este momento, solo ofrecen la posibilidad de comunicarte con ellos a través de llamadas de voz, y a través de sus apps personalizadas, el envío de la ubicación actual del llamante y, si ellos lo piden, de fotografías para una mejor intervención.

	Ante esta situación, los proveedores de PSAPs deberían contemplar la integración de las tecnologías en tiempo real, para recibir no solo voz, sino también texto en tiempo real, ficheros multimedia, videollamadas u otra información relevante. Y, de este modo, proveer un acceso universal, como a ciudadanos con algún tipo de discapacidad.

	A lo largo de este documento, se describe un estudio en profundidad de las tecnologías en tiempo real existentes y el estado actual de las apps de emergencias. En línea con el objetivo de mejorar la conectividad a los servicios de emergencias, este documento también  plantea una solución software fácil de integrar que cumple con los requisitos específicos del 112.
\end{abstractEs}

% Palabras clave en español
\begin{keywordsEs}
	mensajería instantánea, app, centro 112, PSAP
\end{keywordsEs}

% Resumen en inglés
\chapter*{Abstract}

\begin{abstractEn}
	Today, the web tends to be an interactive space where communication between clients and  a web server is bidirectional and information is sending in “real-time”. This change can be reflected in mobile phones, with the instant messaging apps boom that offer a new way of communication by sending of free text messages.

	Many companies have opted to remodel their services to take advantage of the benefits of real-time technologies, because now there are tools that greatly ease this process that was previously a great effort.

	However, the 112 centres (PSAPs) around Europe have not yet taken advantage of these new technology trends. Until now, they only provide the possibility of communicating with them by voice calls, and via their custom apps, sending the caller’s current location and, if they ask, photographs for a better intervention.

	Faced with this situation, PSAP providers should take the integration of real-time technologies into consideration, to receive not only voice, but also real-time text, multimedia files, video calls or other significant information. And, doing so, they would be able to provide universal access, also to citizens with some kind of disability.

	An in-depth study of present real-time technologies and the current stage of emergency apps is described throughout this document. Aligned with the challenge of upgrading the connectivity to emergency services, this project also proposes a software solution easy to integrate that complies with the specific 112 requirements.
\end{abstractEn}

% Palabras clave en inglés
\begin{keywordsEn}
	instant messaging, app, 112 centre, PSAP
\end{keywordsEn}
