% Resumen en español
\chapter*{Resumen}

\begin{abstractEs}
	Hoy en día, la Web tiende a ser un espacio interactivo donde la comunicación entre un cliente y un servidor web es bidireccional y la información se mueve en tiempo real. Este cambio puede reflejarse en los teléfonos móviles, con la explosión de las apps de mensajería instantánea que ofrecen una nueva forma de comunicación a través del envío de mensajes de texto gratuitos.

	Muchas empresas han apostado por remodelar sus servicios para aprovecharse de los beneficios de las tecnologías en tiempo real, debido a que ahora existen herramientas que nos facilitan enormemente este proceso y ya supone un gran esfuerzo.

	Pero los centros 112 de Europa todavía no se han aprovechado de estas nuevas tendencias. Hasta este momento, solo ofrecen la posibilidad de comunicarte con los servicios de emergencias a través de llamadas de voz, y a través de sus apps gratuitas, el envío de la ubicación actual del ciudadano y, si es necesario, de fotografías para agilizar una primera intervención.

	Ante esta situación, los proveedores de PSAPs deberían contemplar la integración de las tecnologías en tiempo real, para recibir no solo voz, sino también texto en tiempo real, ficheros multimedia, videollamadas u otra información relevante. Y de este modo, proveer un acceso universal como a ciudadanos con algún tipo de discapacidad que les impide realizar una llamada al 112 en condiciones.

	A lo largo de este documento, con el objetivo de mejorar la respuesta de los servicios de emergencias, se describe el estudio de nuevas alternativas tecnologías en tiempo real y el desarrollo de una solución software con baja complejidad de integración y cuyo rendimiento cumpla con los requisitos impuestos por los centros 112.

\end{abstractEs}

% Palabras clave en español
\begin{keywordsEs}
	mensajería instantánea, app, Centro 112, PSAP
\end{keywordsEs}

% Resumen en inglés
\chapter*{Abstract}

\begin{abstractEn}
TODO: Resumen en inglés, 250-500 palabras.

\end{abstractEn}

% Palabras clave en inglés
\begin{keywordsEn}
TODO: Palabras clave en inglés, separadas por coma.
\end{keywordsEn}


